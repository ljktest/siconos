

 \paragraph{Newton's linearization of the second  line of~(\ref{eq:toto1})}
The same operation is performed with the second equation of (\ref{eq:toto1})
\begin{equation}
  \begin{array}{l}
    \mathcal R_y(x,y,\lambda)=y-h(t_{k+1},x,\lambda) =0\\ \\
  \end{array}
\end{equation}
which is linearized as
\begin{equation}
  \label{eq:NL9}
  \begin{array}{l}
    \mathcal R_{Ly}(x^{\alpha+1}_{k+1},y^{\alpha+1}_{k+1},\lambda^{\alpha+1}_{k+1}) = \mathcal
    R_{y}(x^{\alpha}_{k+1},y^{\alpha}_{k+1},\lambda^{\alpha}_{k+1}) +
    (y^{\alpha+1}_{k+1}-y^{\alpha}_{k+1})- \\[2mm] \qquad  \qquad \qquad \qquad  \qquad \qquad
    C^{\alpha}_{k+1}(x^{\alpha+1}_{k+1}-x^{\alpha}_{k+1}) - D^{\alpha}_{k+1}(\lambda^{\alpha+1}_{k+1}-\lambda^{\alpha}_{k+1})=0
  \end{array}
\end{equation}

This leads to the following linear equation
\begin{equation}
  \boxed{y^{\alpha+1}_{k+1} =  y^{\alpha}_{k+1}
  -\mathcal R^{\alpha}_{yk+1}+ \\
  C^{\alpha}_{k+1}(x^{\alpha+1}_{k+1}-x^{\alpha}_{k+1}) +
  D^{\alpha}_{k+1}(\lambda^{\alpha+1}_{k+1}-\lambda^{\alpha}_{k+1})}. \label{eq:NL11y}
\end{equation}
with,
\begin{equation}
     \begin{array}{l}
  C^{\alpha}_{k+1} = \nabla_xh(t_{k+1}, x^{\alpha}_{k+1},\lambda^{\alpha}_{k+1} ) \\ \\
  D^{\alpha}_{k+1} = \nabla_{\lambda}h(t_{k+1}, x^{\alpha}_{k+1},\lambda^{\alpha}_{k+1})
 \end{array}
\end{equation}
and
\begin{equation}\fbox{$
\mathcal R^{\alpha}_{yk+1} \stackrel{\Delta}{=} y^{\alpha}_{k+1} - h(x^{\alpha}_{k+1},\lambda^{\alpha}_{k+1})$}
 \end{equation}
 \paragraph{Newton's linearization of the third  line of~(\ref{eq:toto1})}
The same operation is performed with the third equation of (\ref{eq:toto1})
\begin{equation}
  \begin{array}{l}
    \mathcal R_r(r,x,\lambda)=r-g(x,\lambda,t_{k+1}) =0\\ \\  \end{array}
\end{equation}
which is linearized as
\begin{equation}
  \label{eq:NL9}
  \begin{array}{l}
      \mathcal R_{L\lambda}(r^{\alpha+1}_{k+1},x^{\alpha+1}_{k+1},\lambda^{\alpha+1}_{k+1}) = \mathcal
      R_{rk+1}^{\alpha} + (r^{\alpha+1}_{k+1} - r^{\alpha}_{k+1}) -
      K^{\alpha}_{k+1}(x^{\alpha+1}_{k+1} - x^{\alpha}_{k+1})- B^{\alpha}_{k+1}(\lambda^{\alpha+1}_{k+1} -
      \lambda^{\alpha}_{k+1})=0
    \end{array}
  \end{equation}
\begin{equation}
  \label{eq:rrL}
  \begin{array}{l}
    \boxed{r^{\alpha+1}_{k+1} = g(x ^{\alpha}_{k+1},\lambda ^{\alpha}_{k+1},t_{k+1}) -B^{\alpha}_{k+1}
      \lambda^{\alpha}_{k+1} + B^{\alpha}_{k+1} \lambda^{\alpha+1}_{k+1}-K^{\alpha}_{k+1}
      x^{\alpha}_{k+1} + K^{\alpha}_{k+1} x^{\alpha+1}_{k+1}}       
  \end{array}
\end{equation}
with,
\begin{equation}
     \begin{array}{l}
  K^{\alpha}_{k+1} = \nabla_xg(x^{\alpha}_{k+1},\lambda ^{\alpha}_{k+1},t_{k+1})  \\ \\
  B^{\alpha}_{k+1} = \nabla_{\lambda}g(x^{\alpha}_{k+1},\lambda ^{\alpha}_{k+1},t_{k+1})
 \end{array}
\end{equation}
and the  residue for $r$:
\begin{equation}
\boxed{\mathcal
      R_{rk+1}^{\alpha} = r^{\alpha}_{k+1} - g(x^{\alpha}_{k+1},\lambda ^{\alpha}_{k+1},t_{k+1})}
  \end{equation}


\paragraph{Reduction to a linear relation between  $x^{\alpha+1}_{k+1}$ and
$\lambda^{\alpha+1}_{k+1}$}

Inserting (\ref{eq:rrL}) into~(\ref{eq:rfree-12}), we get the following linear relation between $x^{\alpha+1}_{k+1}$ and
$\lambda^{\alpha+1}_{k+1}$, 

\begin{equation}
   \begin{array}{l}
     x^{\alpha+1}_{k+1} = h\gamma(W^{\alpha}_{k+1} )^{-1}\left[g(x^{\alpha}_{k+1},\lambda^{\alpha}_{k+1},t_{k+1}) +
    B^{\alpha}_{k+1} (\lambda^{\alpha+1}_{k+1} - \lambda^{\alpha}_{k+1})+K^{\alpha}_{k+1}
    (x^{\alpha+1}_{k+1} - x^{\alpha}_{k+1}) \right ] +x^\alpha_{free}
\end{array}
\end{equation}
that is 
\begin{equation}
  \begin{array}{l}
    (I-h \gamma (W^{\alpha}_{k+1})^{-1}K^{\alpha}_{k+1})x^{\alpha+1}_{k+1}=x_p + h \gamma (W^{\alpha}_{k+1})^{-1}    B^{\alpha}_{k+1} \lambda^{\alpha+1}_{k+1}
   \end{array}
\end{equation}
with 
\begin{equation}
  \boxed{x_p \stackrel{\Delta}{=}  h\gamma(W^{\alpha}_{k+1} )^{-1}\left[g(x^{\alpha}_{k+1},\lambda^{\alpha}_{k+1},t_{k+1}) +
    -B^{\alpha}_{k+1} (\lambda^{\alpha}_{k+1})-K^{\alpha}_{k+1} (x^{\alpha}_{k+1}) \right ] +x^\alpha_{free}}
\end{equation}



Let us  define the new matrix
\begin{equation}
\hat K^{\alpha}_{k+1}=(I-h \gamma (W^{\alpha}_{k+1})K^{\alpha}_{k+1}).
\label{eq:hatW}
\end{equation}
We get the linear relation
\begin{equation}
  \label{eq:rfree-13}
  \begin{array}{l}
 \boxed{   x^{\alpha+1}_{k+1}\stackrel{\Delta}{=} \hat K^{\alpha,-1}_{k+1} x_p + \hat K^{\alpha,-1}_{k+1} \left[ h \gamma (W^{\alpha}_{k+1})^{-1}    B^{\alpha}_{k+1} \lambda^{\alpha+1}_{k+1}\right]}
   \end{array}
\end{equation}



\paragraph{Reduction to a linear relation between  $y^{\alpha+1}_{k+1}$ and
$\lambda^{\alpha+1}_{k+1}$}

Inserting (\ref{eq:rfree-13}) into (\ref{eq:NL11y}), we get the following linear relation between $y^{\alpha+1}_{k+1}$ and $\lambda^{\alpha+1}_{k+1}$, 
\begin{equation}
   \begin{array}{l}
 y^{\alpha+1}_{k+1} = y_p + \left[ h  C^{\alpha}_{k+1} (\tilde K^{\alpha}_{k+1})^{-1}( W^{\alpha}_{k+1})^{-1}  B^{\alpha}_{k+1} + D^{\alpha}_{k+1} \right]\lambda^{\alpha+1}_{k+1}
   \end{array}
\end{equation}
with 
\begin{equation}\boxed{
y_p = y^{\alpha}_{k+1} -\mathcal R^{\alpha}_{yk+1} + C^{\alpha}_{k+1}(x_q) -
D^{\alpha}_{k+1} \lambda^{\alpha}_{k+1} }
\end{equation}
\textcolor{red}{
  \begin{equation}
   \boxed{ x_q=(\tilde K^{\alpha}_{k+1})^{-1}x_p -x^{\alpha}_{k+1}\label{eq:xqq}}
  \end{equation}
}







% \paragraph{With $\gamma =1$:}
% \[(W^{\alpha}_{k+1} )x^{\alpha+1}_{k+1}= hr^{\alpha+1}_{k+1}- \mathcal R_{free k+1} ^{\alpha}+W^{\alpha}_{k+1}x^{\alpha}_{k+1}\]
% \[x^{\alpha+1}_{k+1}= h( W^{\alpha}_{k+1})^{-1}r^{\alpha+1}_{k+1}-
% ( W^{\alpha}_{k+1})^{-1} \mathcal R_{free k+1} ^{\alpha}+x^{\alpha}_{k+1}\]
% \[x^{\alpha+1}_{k+1}= h( W^{\alpha}_{k+1})^{-1}r^{\alpha+1}_{k+1}+x_{free}\]
% with, using \ref{}
% \begin{equation}
% x_p-x^{\alpha}_{k+1}=h(
% W^{\alpha}_{k+1})^{-1}(g(x^{\alpha}_{k+1},\lambda^{\alpha}_{k+1},t_{k+1})-B^{\alpha}_{k+1}
% \lambda^{\alpha}_{k+1}-K^{\alpha}_{k+1} x^{\alpha}_{k}))+\tilde x_{free}
% \end{equation}
% \[    \tilde x_{free}= -( W^{\alpha}_{k+1})^{-1} \mathcal R _{free k+1} ^{\alpha} \]
%       \[x_{free} = \tilde x_{free} + x^{\alpha}_{k+1}=\fbox{$- W^{-1}R_{free k+1} ^{\alpha} + x^{\alpha}_{k+1}$}\]
% \[ \fbox{$x_p  = x_{free} + h ( W^{\alpha}_{k+1})^{-1}( g(x ^{\alpha}_{k+1},\lambda ^{\alpha}_{k+1},t_{k+1}) -
%       B^{\alpha}_{k+1} \lambda^{\alpha}_{k+1}-K^{\alpha}_{k+1} x^{\alpha}_{k+1} )$} \]




\paragraph{Mixed linear complementarity problem (MLCP)}To summarize, the problem to be solved in each Newton iteration is:\\{
  \begin{minipage}[l]{1.0\linewidth}
    \begin{equation}
      \begin{cases}
      \begin{array}[l]{l}
        y^{\alpha+1}_{k+1} =   W_{mlcpk+1}^{\alpha}  \lambda^{\alpha+1}_{k+1} + b^{\alpha}_{k+1}
        \\ \\
        -y^{\alpha+1}_{k+1} \in N_{[l,u]}(\lambda^{\alpha+1}_{k+1} ). 
      \end{array}
      \label{eq:NL14}
      \end{cases}
    \end{equation}
  \end{minipage}
}
with $W_{mlcpk+1}\in \RR^{m\times m}$ and $b\in\RR^{m}$ defined by
\begin{equation}
  \label{eq:NL15}
 \begin{array}[l]{l}
   W_{mlcpk+1}^{\alpha} = h  C^{\alpha}_{k+1} (\tilde K^{\alpha}_{k+1})^{-1} (W^{\alpha}_{k+1})^{-1}  B^{\alpha}_{k+1} + D^{\alpha}_{k+1} \\
   b^{\alpha}_{k+1} = y_p
\end{array}
\end{equation}

The problem~(\ref{eq:NL14}) is equivalent to a Mixed Linear Complementarity Problem (MLCP) which can be solved under suitable assumptions by many linear complementarity solvers such as pivoting techniques, interior point techniques and splitting/projection strategies. The  reformulation into a standard MLCP follows the same line as for the MCP in the previous section. One obtains,
    \begin{equation}
      \begin{array}[l]{l}
        y^{\alpha+1}_{k+1} =   - W^{\alpha}_{k+1}  \lambda^{\alpha+1}_{k+1} + b^{\alpha}_{k+1}
        \\ \\
        (y^{\alpha+1}_{k+1})_i  = 0 \qquad \textrm{ for } i \in \{ 1..n\}\\[2mm]
        0 \leq  (\lambda^{\alpha+1}_{k+1})_i\perp (y^{\alpha+1}_{k+1})_i \geq 0 \qquad \textrm{ for } i \in \{ n..n+m\}\\
      \end{array}
      \label{eq:MLCP1} 
    \end{equation}



\clearpage


%%% Local Variables: 
%%% mode: latex
%%% TeX-master: "DevNotes"
%%% End: 
