\subsection{Gradiant computaion, case oif NewtonEuler with quaternion}

In the section, $q$ is the quaternion of the dynamical system.

\begin{figure}[h]
  \centering
   
  \input{./Figures/NewtonEulerImpact.pstex_t}
  
  \caption{Impact of one DS.}
  \label{figCase}
\end{figure}

$\nabla _q h$ consist in computing $P_c(\frac{q+\delta q}{\|q+\delta q\|})-P_c(q)$.
\[GP(q)=qG_0P_0~^cq\]
\[GP(\frac{q+\delta q}{\|q+\delta q\|})=(q+\delta q)G_0P_0~^c(q+\delta q)\frac{1}{\|q+\delta q\|^2}\]
\[=(q+\delta q)~^cqGP(q)q~^c(q+\delta q)\frac{1}{\|q+\delta q\|^2}\]
\[=(1,0,0,0)+\delta q~^cq)GP(q)(q~^cq+q~^c\delta q)\frac{1}{\|q+\delta q\|^2}\]
\[=GP(q)+\delta q~^cqGP(q) + GP(q)q~^c\delta q+0(\delta q)^2\frac{1}{\|q+\delta q\|^2}\]
So, because G is independant of $q$:
\[P(\frac{q+\delta q}{\|q+\delta q\|})-P(q)=qGP(\frac{q+\delta q}{\|q+\delta q\|})-GP(q)=\delta q~^cqGP(q) + GP(q)q~^c\delta q+0(\delta q)^2 + GP(q)\frac{1}{\|q+\delta q\|^2}\]
For the directional derivation, we chose $\delta q = \epsilon * (1,0,0,0)$. using a equivalent to $\frac{1}{1+\epsilon}$
\[\lim_{\epsilon \to 0}\frac{P(\frac{q+\delta q}{\|q+\delta q\|})-P(q)}{\epsilon}=~^cqGP(q) + GP(q)q-2q_iGP(q)\]
For the directional derivation, we chose $\delta q = \epsilon * (0,1,0,0)=\epsilon * e_i$
\[\lim_{\epsilon \to 0}\frac{P(\frac{q+\delta q}{\|q+\delta q\|})-P(q)}{\epsilon}=e_i~^cqGP(q) - GP(q)qe_i-2q_iGP(q)\]
Application to the NewtonEulerRImpact:
\[H:\mathbb{R}^7 \to \mathbb{R}\]
\[\nabla _q H \in \mathcal{M}^{1,7}\]
\[\nabla _q H =\left(\begin{array}{c} N_x\\N_y\\N_z\\
(~^cqGP(q) + GP(q)q-2q_0GP(q)).N\\
(e_2~^cqGP(q) - GP(q)qe_2-2q_1GP(q)).N\\
(e_3~^cqGP(q) - GP(q)qe_3-2q_2GP(q)).N\\
(e_4~^cqGP(q) - GP(q)qe_4-2q_3GP(q)).N\\
\end{array}\right)\]
\subsection{Ball case}
It is the case where $GP=-N$:
for $e2$:
\[(0,1,0,0).(q_0,-\underline p).(0,-N)=\]
\[\left(\left(\begin{array}{c}1\\0\\0\end{array}\right).\underline p,\left(\begin{array}{c}q_0\\0\\0\end{array}\right) -\left(\begin{array}{c}1\\0\\0\end{array}\right)*\underline p \right).(0,-N)=\]
\[\left(?, -\underline p_x~N-\left(\left(\begin{array}{c}q_0\\0\\0\end{array}\right)- \left(\begin{array}{c}1\\0\\0\end{array}\right)*\underline p \right)*N\right)=\]
and:
\[(0,-N).(q_0,\underline p).(0,1,0,0)=\]
\[(N.\underline p,-q_0N-N*\underline p).(0,1,0,0)=\]
\[\left(?,(N.\underline p)\left(\begin{array}{c}1\\0\\0\end{array}\right) + \left(\begin{array}{c}1\\0\\0\end{array}\right)*(q_0N+N*\underline p)\right)=\]
\[\left(?,(N.\underline p)\left(\begin{array}{c}1\\0\\0\end{array}\right)+q_0 \left(\begin{array}{c}1\\0\\0\end{array}\right)*N+\left(\begin{array}{c}1\\0\\0\end{array}\right)*(N*\underline p)\right)\]
sub then and get the resulting vector.N:
\[\left[ -\underline p_x~N -N.\underline p~\left(\begin{array}{c}1\\0\\0\end{array}\right)+()*N-\left(\begin{array}{c}1\\0\\0\end{array}\right)*(N*\underline p)\right].N=\]
\[-\underline p_x-N_xN.\underline p+0-(\left(\begin{array}{c}1\\0\\0\end{array}\right)*(N*\underline p)).N=\]
  using $a*(b*c)=b(a.c)-c(a.b)$ leads to
  \[-q_1-N_xN.\underline p-(q_1~N-N_x~\underline p).N=\]
\[-q_1-N_xN.\underline p-q_1+N_xN.\underline p=-2q_1\]
for $e1=(1,0,0,0)$:
\[(q_0,-\underline p).(0,-N)=(?,-q_0N+\underline p*N)\]
\[(0,-N).(q_0,\underline p)=(?,-q_0N-\underline p*N)\]
So
\[\nabla _q H =\left(\begin{array}{c} N_x\\N_y\\N_z\\
0\\
0\\
0\\
0\\
\end{array}\right)\]


\subsection{Case NewtonLaw : using the local frame and momentum}

\[\left(\begin{array}{c}m \dot V\\I \dot \omega + \omega I \omega \end{array}\right)= \left(\begin{array}{c}Fect+R\\Mext + R*PG \end{array}\right)\]
  with * vectoriel product, $R$ reaction in the globla frame. $P$ the point of contact.
  $r$ is the reaction in the local frame.  $M^t r=R$ with:
  \[M^t=\left(\begin{array}{c} nx \\ny\\nz \end{array}\right)\]
  with :
  \[\left(\begin{array}{c} R_x \\R_y\\R_z \end{array}\right)*\left(\begin{array}{c} PG_x \\PG_y\\PG_z \end{array}\right) =\left(\begin{array}{c} R_y~PG_z - R_z~PG_y \\R_z~PG_x-R_x~PG_z \\R_x~PG_y-R_y~PG_x \end{array}\right)\]

  we have :
  
  \[\left(\begin{array}{c}R\\R*PG\end{array}\right)=\left(\begin{array}{ccc} 1&0&0\\0&1&0\\0&0&1\\
      0&PG_z&-PG_y\\-PG_z&0&PG_x\\PG_y&-PG_X&0\end{array}\right).R:=N^tR=N^tM^tr\]
      we want:
      
\[\left(\begin{array}{c}m \dot V\\I \dot \omega + \omega I \omega \end{array}\right)=jachqT^t r\]
So $jachqt=MN$

\subsection{Case FC3D: using the local frame and momentum}

\[\left(\begin{array}{c}m \dot V\\I \dot \Omega + \Omega I \Omega \end{array}\right)= \left(\begin{array}{c}Fect+R\\Mext _{R_{obj}} + (R*PG) _{R_{obj}} \end{array}\right)\]
  with * vectoriel product, $R$ reaction in the globla frame. $P$ the point of contact.
  $r$ is the reaction in the local frame.  $M_{R_{obj}toR_{abs}}=M_{R_{abs}toR_{obj}}^t r=R$ with:
  \[M_{R_{obj}toR_{abs}}=\left(\begin{array}{ccc} nx&t_1x&t_2x \\ny&t_1y&t_2y\\nz&t_1z&t_2z \end{array}\right)\]
  we have :
  \[\left(\begin{array}{c}R\\(R*PG) _{R_{obj}}\end{array}\right)=\left(\begin{array}{c} I_3\\N_{PG}\end{array}\right).R:=N^tR=N^tM^tr\]
  \[ N_{PG}=\left(\begin{array}{ccc} 0&PG_z&-PG_y\\-PG_z&0&PG_x\\PG_y&-PG_X&0\end{array}\right)\]
      we want:
      
\[\left(\begin{array}{c}m \dot V\\I \dot \Omega + \Omega I \Omega \end{array}\right)=jachqT^t r\]
So $jachqt=MN$

\subsection{Case FC3D: using the local frame local velocities}
\begin{figure}[h]
  \centering
   \scalebox{0.7}{
  \input{./Figures/SolideContact.pstex_t}
  }
  \caption{Two objects colliding.}
  \label{figCase}
\end{figure}


We are looking for an operator named $CT$ such that:

\[V_C=\left(\begin{array}{c} V_N \\ V_T \\ V_S \end{array}\right)_{R_{C}}=CT \left(\begin{array}{c} V_{G1}~_{R_{abs}} \\ \Omega_1~_{R_{obj1}} \\ V_{G2}~_{R_{abs}}\\ \Omega_2~_{R_{obj2}} \end{array}\right)\]

\[V_c=V_{G1}~_{R_{abs}} + w_1 * G_1P~_{R_{abs}} -(V_{G2}~_{R_{abs}} + w_2 * G_1P~_{R_{abs}})\]
where $w_1$ and $w_2$ are given in $R_{abs}$. We note $M_{R_{obj1}toR_{abs}}$ the matrice converting the object 1 coordinate to the absolute coordinate. We note $N_1$ the matrice such that $w_1*G_1P~_{R_{abs}} = N_1 w_1$. Endly, we note $M_{R_{abs}toR_C}$ converting the absolute coordinate to the $R_C$ frame.
we get:
\[CT= M_{R_{abs}toR_C}   \left(\begin{array}{cccc} I_3 & N_1M_{R_{obj1}toR_{abs}} & -I_3 & -N_2M_{R_{obj2}toR_{abs}} \end{array}\right)\]

\subsubsection{Expression of $M_{R_{obj1}toR_{abs}}$}
Using quaternion, we get :
\[M_{R_{obj1}toR_{abs}} = \left(\begin{array}{ccc} q \left(\begin{array}{c}1\\0\\0 \end{array}\right)~^cq & q \left(\begin{array}{c}  0\\1\\0 \end{array}\right)~ ^cq & q \left(\begin{array}{c}  0\\0\\1 \end{array}\right)~ ^cq  \end{array}\right)\]

\subsubsection{Expression of $N_1$}
\[N_1=\left(\begin{array}{ccc} 0&G_1C_z&-G_1C_y\\-G_1C_z&0&G_1C_x\\G_1C_y&-G_1C_X&0\end{array}\right)\]

