This section is about the Newton-Euler equations:

\subsection{The dynamical system}

\begin{equation}
\label{NE_Dyn1}
\begin{array}{l}
m \dot{V}(t) = F_{ext}  \\
I \dot \Omega +\Omega \wedge I\Omega = M_{ext}
\end{array}
%\caption{Newton-Euler dynamical equations}
\end{equation}

Where $V$ is the velocity of the center of mass and $\Omega$ is the angular speed in the
referential attached to the object.

\subsection{The relation}
X is the position and the orientation, we don't focus on its representation.
\begin{equation}
\label{Relation}
\begin{array}{l}
Y=H(X)  \\
R=G(X,\lambda)
\end{array}
\end{equation}


The first equation is derived:
\[\dot Y = C \dot X + \dot H\]
It exists an operator T :
\[T:  \left(\begin{array}{l} V\\ \Omega\end{array}\right) \to \dot X \]
  Using T leads to :
\[\dot Y = C T \left(\begin{array}{l} V\\ \Omega\end{array}\right) + \dot H\]

\subsection{discretization $t_k \to t_{k+1}$}
The goal of this section is to describe the computation done in Siconos.
\subsubsection{explicit case}
The dynamical system~\ref{NE_Dyn1} leads to the system:
\begin{equation}
  \left(\begin{array}{cc} m&0\\0&I\end{array}\right)
   \left(\begin{array}{c} V_{k+1} - V_k\\ \Omega _{k+1} - \Omega _{k}\end{array}\right)=
   h\left(\begin{array}{c} Fext_k\\ Mext_k - \Omega _k \wedge I\Omega _k \end{array}\right)+
   h B \lambda _{k+1}
  \end{equation}
Using $W = \left(\begin{array}{cc} m&0\\0&I\end{array}\right) ^{-1} $
\begin{equation}
  \label{NE_dis_explicit}
  \fbox{$
   \left(\begin{array}{c} V_{k+1} \\ \Omega _{k+1}\end{array}\right)=
   hW\left(\begin{array}{c} Fext_k\\ Mext_k - \Omega _k \wedge I\Omega _k \end{array}\right)+
   hWB \lambda _{k+1}+
   \left(\begin{array}{c} V_{k} \\ \Omega _{k}\end{array}\right)
   $}
  \end{equation}

The relation~\ref{Relation} leads to the system:

\[\dot Y _{k+1}= C T \left(\begin{array}{l} V_{k+1} \\ \Omega _{k+1} \end{array}\right) + \dot H _{k} \]
Substitute $  \left(\begin{array}{l} V_{k+1} \\ \Omega _{k+1} \end{array}\right) $
using~\ref{NE_dis_explicit} leads:

\[\dot Y _{k+1}= C T \lbrack hW\left(\begin{array}{c} Fext_k\\ Mext_k - \Omega _k \wedge I\Omega _k \end{array}\right)+
   hWB \lambda _{k+1}+
   \left(\begin{array}{c} V_{k} \\ \Omega _{k}\end{array}\right) \rbrack
   + \dot H _{k} \]
Ones gets:
\[\fbox{$
\dot Y _{k+1}= hC T W B \lambda _{k+1} +hCTW \left(\begin{array}{c} Fext_k\\ Mext_k - \Omega _k
   \wedge I\Omega _k \end{array}\right) + CT \left(\begin{array}{c} V_{k} \\ \Omega
   _{k}\end{array}\right) +\dot H _{k} $}\]


Solving the one step problem gives $\lambda _{k+1}$, from~\ref{NE_dis_explicit} we get
$ \left(\begin{array}{l} V_{k+1} \\ \Omega _{k+1} \end{array}\right) $. $V$ and $\Omega$ are integrated
to get $X_{k+1}$.

\paragraph{Implementation in Siconos:}
\begin{enumerate}
  \item[--] The Moreau::comouteResiduFree computes ds->residuFree = -FL that is
  $-\left(\begin{array}{c} Fext_k\\ Mext_k - \Omega _k \wedge I\Omega _k \end{array}\right) $ thanks
  ds::computeFL
  \item[--] The Moreau::computeFreeState computes $X_{free} = T(V_i^{k+1} - hW (ResiduFree))$
    \item[--] LinearOSNS::computeQBlock computes $q=C X_{free}$
  \end{enumerate}

\subsection{Quaternion case}
We chose $X= \left(\begin{array}{l} X_g \\q \end{array}\right) $. $X_g$ are the coordinates of the center of mass, and $q$ is a quaternion
  represented the orientation of solid. It means :
  \[q_k(0,GM_0)q_k^c = (0,GM_k)\]
Where G is the center of mass, and M any point of the solid.\\
This section describes the $T$ operator in this case. Computation using quaternion leads to the relation:
\[\dot q = \frac{1}{2} q (0,\Omega)\]
So using the matrix formulation:
\[\dot q = \frac{1}{2}  \left(\begin{array}{cccc} q_0&-q_1&-q_2&-q_3 \\ q_1&q_0&-q_3&q_2\\
  q_2&q_3&q_0&-q_1\\ q_3&-q_2&-q_1&q_0\end{array}\right)  \left(\begin{array}{c} 0 \\ \Omega
  \end{array}\right) =
  T_q   \Omega  \]
  That lead to :
  \[ \dot X = \left(\begin{array}{cc} I_3 & 0 \\ 0 &
  T_q \end{array}\right) \left(\begin{array}{c} V\\ \Omega  \end{array}\right)  = T \left(\begin{array}{c} V\\ \Omega  \end{array}\right)\]

It is noteworthy that $T$ must be updated at each step.

