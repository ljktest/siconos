This section is about the Newton-Euler equations:

\subsection{The dynamical system}

\begin{equation}
\label{NE_Dyn1}
\begin{array}{l}
m \dot{V}(t) = F_{ext}  \\
I \dot \Omega +\Omega \wedge I\Omega = M_{ext}
\end{array}
%\caption{Newton-Euler dynamical equations}
\end{equation}

Where $V$ is the velocity of the center of mass and $\Omega$ is the angular speed in the
referential attached to the object.

\subsection{The relation}
X is the position and the orientation, we don't focus on its representation.
\begin{equation}
\label{Relation}
\begin{array}{l}
Y=H(X)  \\
R=G(X,\lambda)
\end{array}
\end{equation}


The first equation is derived:
\[\dot Y = C \dot X + \dot H\]
It exists an operator T :
\[T:  \left(\begin{array}{l} V\\ \Omega\end{array}\right) \to \dot X \]
  Using T leads to :
\[\dot Y = C T \left(\begin{array}{l} V\\ \Omega\end{array}\right) + \dot H\]

\subsection{discretization $t_k \to t_{k+1}$, and implementation in Siconos}
The goal of this section is to describe the computation done in Siconos. The unknown are rename
using the Siconos convention.
\subsubsection{the unknowns}
About the DS:
\[\_v_{k}=\left(\begin{array}{c} V_k\\ \Omega _{k}\end{array}\right)\]
And $\_q_{k}$ represent the system, usually it could be the coordinate of the center of mass and a
representation of the orientation.

\subsubsection{explicit case}
The dynamical system~\ref{NE_Dyn1} leads to the system:
\begin{equation}
  \left(\begin{array}{cc} m&0\\0&I\end{array}\right)
   (\_v_{k+1}-\_v_{k})=
   h \_Fl_k +
    ^t(CT) h\lambda _{k+1}
  \end{equation}
  With \[\_Fl_k = \left(\begin{array}{c} Fext_k\\ Mext_k - \Omega _k \wedge I\Omega _k \end{array}\right)\]
Using $W = \left(\begin{array}{cc} m&0\\0&I\end{array}\right) ^{-1} $
\begin{equation}
  \label{NE_dis_explicit}
  \fbox{$
   \_v_{k+1}=
   W (h \_Fl_k)+
   W ^t(CT)h\lambda _{k+1}+ \_v_k
   $}
  \end{equation}
  This computation is done in Moreau::updateState, using:
  \[\_ResiduFree_k = -h \_Fl_k\]
  \[Xfree_k = -W \_ResiduFree_k + \_v_k\]

The relation~\ref{Relation} leads to the system:

\[\dot Y _{k+1}= C T \_v_{k+1} + \dot H _{k} \]
Substitute $  \_v_{k+1} $ using~\ref{NE_dis_explicit} leads:

\[\dot Y _{k+1}= C T \lbrack W h\_Fl_k+
   hW^t(CT) \lambda _{k+1}+ \_v_k \rbrack
   + \dot H _{k} = C T \lbrack  hW ^t(CT) \lambda _{k+1}+ Xfree_k \rbrack
   + \dot H _{k}\]
Ones gets:
\[\fbox{$
\dot Y _{k+1}= C T W ^t(CT) (h\lambda _{k+1}) + CT Xfree_k +\dot H _{k} $}\]


Solving the one step problem gives $h\lambda _{k+1}$, and from~\ref{NE_dis_explicit} we get
$ \_v_{k+1} $. At least, $ \_v_{k+1} $ is used to compute $\dot \_q_{k+1}$, provided $\_q_{k+1}$.

\subsection{Quaternion case}
Working in 3D, we chose $\_q= \left(\begin{array}{l} X_g \\q \end{array}\right) $. $X_g$ are the 3 coordinates of the center of mass, and $q$ is a quaternion
  represented the orientation of solid. It means :
  \[q_k(0,GM_0)q_k^c = (0,GM_k)\]
Where G is the center of mass, and M any point of the solid.\\
This section describes the $T$ operator in this case. Computation using quaternion leads to the relation:
\[\dot q = \frac{1}{2} q (0,\Omega)\]
So using the matrix formulation:
\[\dot q = \frac{1}{2}  \left(\begin{array}{cccc} q_0&-q_1&-q_2&-q_3 \\ q_1&q_0&-q_3&q_2\\
  q_2&q_3&q_0&-q_1\\ q_3&-q_2&-q_1&q_0\end{array}\right)  \left(\begin{array}{c} 0 \\ \Omega
  \end{array}\right) =
  T_q   \Omega  \]
  That lead to :
  \[ \dot \_q = \left(\begin{array}{cc} I_3 & 0 \\ 0 &
  T_q \end{array}\right) \left(\begin{array}{c} V\\ \Omega  \end{array}\right)  = T
  \left(\begin{array}{c} V\\ \Omega  \end{array}\right)=T \_v\]

It is noteworthy that $T$ must be updated at each step.

\subsection{Case of the Relation between many DS}
We focus on relation using 2 DS:
\[Y=H(\_q1,\_q2)\]
