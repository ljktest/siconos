


\usepackage{a4wide}
\usepackage{amsmath}
\usepackage{amssymb}
\usepackage{minitoc}
%\usepackage{glosstex}
\usepackage{colortbl}
\usepackage{hhline}
\usepackage{longtable}

\def\glossaryname{Glossary of Notation}



\usepackage{color}
\usepackage{graphicx,epsfig}
\graphicspath{{figure/}}
\usepackage[T1]{fontenc}
\usepackage{rotating}

\usepackage{algorithmic}
\usepackage{algorithm}
%\usepackage{ntheorem}
\usepackage{natbib}

\renewcommand{\algorithmiccomment}[1]{//#1}

%\renewcommand{\baselinestretch}{2.0}
\setcounter{tocdepth}{2}     % Dans la table des matieres
\setcounter{secnumdepth}{3}  % Avec un numero.



%\newtheorem{definition}{Definition}
%\newtheorem{lemma}{Lemma}
%\newtheorem{claim}{Claim}
%\newtheorem{remark}{Remark}
%\newtheorem{assumption}{Assumption}
%\newtheorem{example}{Example}
%\newtheorem{conjecture}{Conjecture}
%\newtheorem{corollary}{Corollary}
%\newtheorem{OP}{OP}
%\newtheorem{problem}{Problem}
%\newtheorem{theorem}{Theorem}


\newcommand{\CC}{\mbox{\rm $~\vrule height6.6pt width0.5pt depth0.25pt\!\!$C}}
\newcommand{\ZZ}{\mbox{\rm \lower0.3pt\hbox{$\angle\!\!\!$}Z}}
\newcommand{\RR}{\mbox{\rm $I\!\!R$}}
\newcommand{\NN}{\mbox{\rm $I\!\!N$}}

\newcommand{\Frac}[2]{\displaystyle \frac{#1}{#2}}

\newcommand{\DP}[2]{\displaystyle \frac{\partial {#1}}{\partial {#2}}}


\newcommand{\ie}{i.e.}
\newcommand{\eg}{e.g.}
\newcommand{\cf}{c.f.}
\newcommand{\putidx}[1]{\index{#1}\textit{#1}}

\def\Er{{\rm I\! R}}
\def\En{{\rm I\! N}} 
\def\Ec{{\rm I\! C}}
 






%----------------------------------------------------------------------
%                  Modification des subsubsections
%----------------------------------------------------------------------
\makeatletter
\renewcommand\thesubsubsection{\thesubsection.\@alph\c@subsubsection}
\makeatother

%----------------------------------------------------------------------
%             Redaction note environnement
%----------------------------------------------------------------------
\makeatletter
%\theoremheaderfont{\scshape}
%\theoremstyle{marginbreak}
%\theorembodyfont{\upshape}
%\newtheorem{rque}{\bf Remarque}[chapter]
%\newtheorem{rque1}{\bf \fsc{Remarque}}[chapter] !!! \fsc est une commande french
%\newtheorem{ndr1}{\textbf{\textsc{Redaction note}}}[section]

\newenvironment{ndr}%
{%
\tt
%\centerline{---oOo---}
\noindent\begin{ndr1}%
}%
{%
\begin{flushright}%
%\vspace{-1.5em}\ding{111}
\end{flushright}%
\end{ndr1}%
%\centerline{---oOo---}
}

\makeatother

%----------------------------------------------------------------------
%             Redaction note environnement V.ACARY
%----------------------------------------------------------------------
\makeatletter
%\theoremheaderfont{\scshape}
%\theoremstyle{marginbreak}
%\theorembodyfont{\upshape}
%\newtheorem{rque}{\bf Remarque}[chapter]
%\newtheorem{rque1}{\bf \fsc{Remarque}}[chapter] !!! \fsc est une commande french
%\newtheorem{ndr1va}{\textbf{\textsc{Redaction note V. ACARY}}}[section]

\newenvironment{ndrva}%
{%
\tt
%\centerline{---oOo---}
\noindent\begin{ndr1va}%
}%
{%
\begin{flushright}%
%\vspace{-1.5em}\ding{111}
\end{flushright}%
\end{ndr1va}%
%\centerline{---oOo---}
}

\makeatother




%%% Local Variables: 
%%% mode: latex
%%% TeX-master: "report"
%%% End: 
