\section{Basic elements of  Lie groups and Lie algebras theory.}
Let us recall the definitions of the  Lie group Theory taken from \cite{Iserles.ea_AN2000} and \cite{Varadarajan_book1984}.




\subsection{Differential equation (evolving) on a manifold $\mathcal M$}
\begin{definition}
 A $d$-dimensional manifold $\mathcal M$ is a $d$-dimensional smooth surface $ M\subset \RR^n$ for some $n\geq d$.
\end{definition}
\begin{definition}
  Let $\mathcal M$ be a $d$-dimensional manifold and suppose that $\rho(t) \in\mathcal M$  is a smooth curve such that $\rho(0) = p$. A tangent vector at $p$ is defined as
  \begin{equation}
    \label{eq:12}
    a = \left. \frac{d}{dt} (\rho(t)) \right|_{t=0}.
  \end{equation}
The set of all tangents at $p$ is called the tangent space at $p$ and denoted by $T\mathcal M|_p$. It has the structure of a linear space. 
\end{definition}
\begin{definition}
   A (tangent) vector field on $\mathcal M$ is a smooth function $F : \mathcal M \rightarrow T\mathcal M$ such that $F (p) \in T\mathcal M|_p$ for all $p \in \mathcal M$. The collection of all vector fields on $\mathcal M$ is denoted by $\mathcal X(\mathcal M)$.
 \end{definition}


 \begin{definition}[Differential equation (evolving) on $\mathcal M$]
   Let $F$ be a tangent vector field on $\mathcal M$. By a differential equation (evolving) on $\mathcal M$ we mean a differential equation of the form
   \begin{equation}
     \dot y =F(y), t\geq  0, y(0)\in \mathcal M\label{eq:13}
   \end{equation}
   where $F \in \mathcal X(\mathcal M)$. Whenever convenient, we allow $F$ in~\eqref{eq:13} to be a function of time, $F = F(t,y)$. The flow of $F$ is the solution operator $\Psi_{t,F} : \mathcal M \rightarrow  \mathcal M$ such that
   \begin{equation}
     y(t) = \Psi_{t,F} (y0).\label{eq:14}
   \end{equation}
 \end{definition}

 \subsection{Lie algebra and Lie group}
 \begin{definition}[commutator]
   Given two vector fields $F, G$ on $\RR^n$ , the commutator $H = [F, G]$ can
   be computed componentwise at a given point $y ∈ \RR^n$ as
   \begin{equation}
     H_i(y)= \sum_{j=1}^n  G_j(y)\frac{\partial F_i(y)}{\partial y_j}   −F_j(y) \frac{\partial G_i(y)}{\partial y_j} .\label{eq:15}
   \end{equation}
 \end{definition}

 \begin{lemma}\label{lemma:LieBracket}
The commutator of vector fields satisfies the identities
\begin{equation}
  \label{eq:16}
  \begin{array}[lclr]{lclr}
    \protect{[}F, G]&=& −\protect{[}G, F ] & (skew symmetry), \\
    \protect{[} \alpha F,G] &=& \alpha \protect{[}F,G], \alpha \in \RR &  \\
    \protect{[}F + G, H]&=& \protect{[}F, H] + \protect{[}G, H] & (bilinearity),\\
    0 &=&  \protect{[}F,\protect{[}G,H]]+\protect{[}G,\protect{[}H,F]]+\protect{[}H,\protect{[}F,G]] &(Jacobi’s identity).
  \end{array}
\end{equation}
\end{lemma}
\begin{definition}
  A Lie algebra of vector fields is a collection of vector fields which is closed under linear combination and commutation. In other words, letting $\mathfrak g$ denote the Lie algebra,
  
  \begin{equation}
    \begin{array}[lclr]{l}
    B \in \mathfrak g \implies \alpha B \in \mathfrak  g \text{ for all } \alpha ∈ R .\\
    B_1,B_2 \in\mathfrak g \implies B_1 +B_2, [B_1,B_2]\in\mathfrak g\label{eq:17}
    \end{array}
\end{equation}

Given a collection of vector fields $B = {B_1 , B_2 , \ldots}$, the least Lie algebra of vector fields containing $B$ is called the Lie algebra generated by $B$
\end{definition}


\begin{definition}
  A Lie algebra is a linear space $V$ equipped with a Lie bracket, a bilinear, skew-symmetric mapping
  \begin{equation}
    \label{eq:18}
    [ \cdot , \cdot ] : V \times V \rightarrow V 
  \end{equation}
that obeys identities \eqref{eq:16} from Lemma~\ref{lemma:LieBracket}
\end{definition}

\begin{definition}[(General) Lie algebra]
  A Lie algebra homomorphism is a linear map between two Lie algebras, $\varphi : \mathfrak g \rightarrow \mathfrak h$, satisfying the identity
  \begin{equation}
\varphi ([v, w]_{\mathfrak g}) = [\varphi(v), \varphi(w)]_{\mathfrak h}, v, w in \mathfrak g\label{eq:19}.
\end{equation}
An invertible homomorphism is called an isomorphism.
\end{definition}

\begin{definition}
  A Lie group is a differential manifold $\mathcal G$ equipped with a product $\glaw : \mathcal G\times \mathcal G →\rightarrow \mathcal  G$ satisfying
  \begin{equation}
    \label{eq:20}
    \begin{array}[lclr]{lr}
      p \glaw(q \glaw r) = (p\glaw q)\glaw r, \forall  p, q, r ∈ \mathcal G &\text{(associativity)}\\
      \exists I \in \mathcal G \text{ such that } I\glaw p = p \glaw I = p,  \forall p \in \mathcal G&\text{(identity element)}\ \\
      \forall p \in \mathcal G, \exists  p^{-1}  \in \mathcal G \text{ such that }  p^{-1}\glaw p = I&\text{(inverse) }\ \\
      \text{ The maps}  (p, r)  \rightarrow p\glaw r \text{ and }  p  \rightarrow p^{-1} \text{are smooth functions }&\text{(smoothness)}\                                                                                                
    \end{array}
\end{equation}
\end{definition}

\begin{definition}[Lie algebra $\mathfrak g $ of a Lie group $\mathcal G$]
  The Lie algebra $\mathfrak g$ of a Lie group $\mathcal G$ is defined as the linear space of all tangents to $G$ at the identity $I$. The Lie bracket in $\mathfrak g$ is defined as
  \begin{equation}
    [a,b]= \left.\frac{\partial^2 }{\partial s\partial t} \rho(s)\sigma(t)\rho(-s)\right|_{s=t=0}\label{eq:21}
\end{equation}
where $\rho(s)$ and $\sigma(t)$ are two smooth curves on $\mathcal G$ such that $\rho(0) = \sigma(0) = I$, and 
$\dot \rho(0) = a$ and $\dot \sigma(0) = b$.
\end{definition}

\subsection{Actions of a group $\mathcal G$ on  manifold $\mathcal M$}
\begin{definition}
   A left  action of Lie Group $\mathcal G$ on a manifold $\mathcal M$ is a smooth map $\Lambda^l: \mathcal G \times  \mathcal M \rightarrow \mathcal M$ satisfying
\begin{equation}
  \label{eq:22}
  \begin{array}[lcl]{rcl}
    \Lambda^l(I,y) &=& y, \quad \forall y \in \mathcal M \\
    \Lambda^l(p,\Lambda(r,y)) &=& \Lambda^l(p\glaw r, y) , \quad \forall p,r \in \mathcal G,\quad  \forall y \in \mathcal M .
  \end{array}
\end{equation}
\end{definition}

\begin{definition}
   A  right  action of Lie Group $\mathcal G$ on a manifold $\mathcal M$ is a smooth map $\Lambda^r: \mathcal M \times \mathcal G   \rightarrow \mathcal M$ satisfying
\begin{equation}
  \label{eq:23}
  \begin{array}[lcl]{rcl}
    \Lambda^r(y,I) &=& y, \quad \forall y \in \mathcal M \\
    \Lambda^r(\Lambda(y,r), p) &=& \Lambda^r(y,  r\glaw p) , \quad \forall p,r \in \mathcal G,\quad  \forall y \in \mathcal M .
  \end{array}
\end{equation}
\end{definition}

A given smooth curve  $S(\cdot) : t\in \RR \mapsto S(t)\in \mathcal G$ in $\mathcal G$ such that $S(0)= I$ produces a flow $\Lambda^l(S(t),\cdot)$ (resp. $\Lambda^r(\cdot, S(t))$) on $\mathcal M$ and by differentiation we find a tangent vector field
\begin{equation}
  \label{eq:24}
  F(y) = \left. \frac{d}{dt} (\Lambda^l(S(t),y) \right|_{t=0}\quad( \text{resp.  }  F(y) = \left. \frac{d}{dt} (\Lambda^r(y,S(t)) \right|_{t=0} ) 
\end{equation}
that defines a ordinary differential equation on a Lie Group
\begin{equation}
  \label{eq:25}
  \dot y(t) = F(y(t)) = \left. \frac{d}{dt} (\Lambda^l(S(t),y) \right|_{t=0}  \quad( \text{resp.  }\dot y(t) = F(y(t)) = \left. \frac{d}{dt} (\Lambda^r(y,S(t)) \right|_{t=0})
\end{equation}
  
\begin{lemma}
  Let $\lambda^l_{*} : \mathfrak g \rightarrow \mathcal X(\mathcal M) $ (resp. $\lambda^r_{*} : \mathfrak g \rightarrow \mathcal X(\mathcal M) $ be defined as
  \begin{equation}
  \lambda^l_{*}(a)(y) = \left.\frac{d}{ds}{ \Lambda^l (\rho(s), y)}\right|_{s=0} \quad (\text{ resp. }  \lambda^r_{*}(a)(y) = \left.\frac{d}{ds}{ \Lambda^r (y, \rho(s))}\right|_{s=0})\label{eq:26}  
\end{equation}
 where $\rho(s)$ is a curve in $\mathcal G$ such that $\rho(0)=I$ and $\dot\rho (0)=a$. Then $\lambda^l_{8}$ is a linear
map between Lie algebras such that
\begin{equation}
  [a, b]_{\mathfrak g} = [\lambda^l_{*}(a), \lambda^l_{*}(b)]_{\mathcal X(\mathcal M)}.\label{eq:27}
\end{equation}
\end{lemma}


The following product between an element of an algebra $a \in \mathfrak g$ with an element of a group $\sigma  \in \mathcal G$ 
 can be defined. This will served as a basis for defining the exponential map.
\begin{definition}
  We define the left product $(\cdot, \cdot)^l : \mathfrak g \times \mathcal G \rightarrow  \mathcal G$ of an element of an algebra $a \in \mathfrak g$ with an element of a group $\sigma  \in \mathcal G$ as
  \begin{equation}
 (a, \sigma)^l = a \cdot \sigma = \left.\frac{d}{ds} \rho(s) \glaw \sigma \right|_{s=0}\label{eq:28}
\end{equation}
where $\rho(s)$ is a smooth curve such that $\dot\rho(0)=a$ and $\rho(0)=I$. In the same way, we can define the right product $(\cdot, \cdot)^r : \mathcal G \times \mathfrak g  \rightarrow   \mathcal G$ 
\begin{equation}
  \label{eq:29}
  (\sigma,a)^r = \sigma \cdot a  = \left.\frac{d}{ds} \sigma \glaw \rho(s)   \right|_{s=0}
\end{equation}
\end{definition}

\subsection{Exponential map}
\begin{definition}
  Let $\mathcal G$ be a Lie group and $\mathfrak g$ its Lie algebra. The exponential mapping $exp : \mathfrak g \rightarrow \mathcal G$ is defined as $\exp(a) = \sigma(1)$ where $\sigma (t)$ satisfies the  differential equation
\begin{equation}
\dot \sigma(t) = a \cdot \sigma(t), \quad \sigma (0) = I.\label{eq:30}
\end{equation}
\end{definition}

Let us define $a^k$ as
\begin{equation}
  \label{eq:31}
  \left\{\begin{array}[l]{l}
    a^k = \underbrace{a\glaw a \glaw \ldots a\glaw a}_{k \text{ times}} \text{ for } k \geq 1 \\
    a^0  = I
  \end{array}\right.
\end{equation}
The exponential map can be expressed as
\begin{equation}
  \label{eq:32}
  \exp(at) = \sum_{k=0}^\infty \frac{(ta)^k}{k!}
\end{equation}
since it is  a solution of \eqref{eq:30}. A simple computation allows to check this claim:
\begin{equation}
  \label{eq:33}
   \frac{d}{dt}\exp(at) = \sum_{k=1}^\infty  k t^{k-1} \frac{a^k}{k!} = a \glaw \sum_{k=0}^\infty  t^{k} \frac{a^k}{k!} = a \glaw \exp(at).
\end{equation}
A similar computation gives
\begin{equation}
  \label{eq:34}
  \frac{d}{dt}\exp(at)  = \sum_{k=0}^\infty  t^{k} \frac{a^k}{k!} \glaw a = \exp(at) \glaw a.
\end{equation}
The exponential mapping $exp : \mathfrak g \rightarrow \mathcal G$ can also be defined as $\exp(a) = \sigma(1)$ where $\sigma (t)$ satisfies the  differential equation
\begin{equation}
  \label{eq:35}
  \dot \sigma(t) = \sigma(t) \cdot a, \quad \sigma (0) = I.
\end{equation}

\begin{theorem}
  \label{Theorem:solutionofLieODE}
  Let $\Lambda^l:\mathcal G\times\mathcal M \rightarrow \mathcal M$ be a left  group action and $\lambda^l_{∗} : \mathfrak g\rightarrow \mathcal X(\mathcal M)$ the corresponding Lie algebra homomorphism. For any $a \in \mathfrak g$ the flow of the vector field $F = \lambda^l_{a}(a)$, i.e. the solution of the equation
  \begin{equation}
    \dot y(t) = F(y(t)) = \lambda^l_{*}(a)(y(t)),\quad  t \geq 0, y(0) = y_0 \in \mathcal M,\label{eq:36}
\end{equation}
  is given as
  \begin{equation}
y(t) = \Lambda^l(\exp(ta), y_0).\label{eq:37}
\end{equation}
Let $\Lambda^r:\mathcal M\times\mathcal G \rightarrow \mathcal M$ be a right group action and $\lambda^r_{∗} : \mathfrak g\rightarrow \mathcal X(\mathcal M)$ the corresponding Lie algebra homomorphism. For any $a \in \mathfrak g$ the flow of the vector field $F = \lambda^r_{*}(a)$, i.e. the solution of the equation
  \begin{equation}
    \dot y(t) = F(y(t)) = \lambda^r_{*}(a)(y(t)),\quad  t \geq 0, y(0) = y_0 \in \mathcal M,\label{eq:38}
\end{equation}
  is given as
  \begin{equation}
y(t) = \Lambda^r(y_0,\exp(ta)).\label{eq:39}
\end{equation}

\end{theorem}


\subsection{Translation (Trivialization) maps}
The left and right translation maps defined by 
\begin{equation}
  \label{eq:148}
  \begin{array}{rcl}
    L_z  : \mathcal G \times \mathcal G &\rightarrow& \mathcal G \quad \text{ (left translation map )} \\
    y &\mapsto&  z \glaw y
  \end{array}
\end{equation}
and 
\begin{equation}
  \label{eq:149}
  \begin{array}{rcl}
    R_z(y)  :  \mathcal G \times  \mathcal G  & \rightarrow& \mathcal G \quad \text{ (right translation map )} \\
    y  &\mapsto&  y \glaw z 
  \end{array}
\end{equation}

If we identify the manifold $\mathcal M$ with the group $\mathcal G$, The left and right translations can be interpreted as the simplest example of group action on the manifold. Note that the left translation map can be viewed as a left or right action on the group.

If we consider $L_z(y)$ as a right group action $ L_z(y) = \Lambda^r( z, y) =z \glaw y $, by differentiation we get a $L'_z : T \mathfrak g \cong  \mathfrak g \rightarrow T_z\mathcal G$ with $\dot\rho (0)=a$ such that
\begin{equation}
  \label{eq:150}
  \lambda^r_{*}(a)(z) = L'_z(a) = \left.\frac{d}{ds}{ \Lambda^r (z, \rho(s))}\right|_{s=0} = z \glaw a
\end{equation}
The map
\begin{equation}
  \label{eq:152}
  \begin{array}{rcl}
  L'_z  : \mathfrak g &\rightarrow& T_z\mathcal G  \\
         a &\mapsto&  z \glaw a
  \end{array}
\end{equation}
determines an isomorphism of $\mathfrak g$ with the tangent space  $T_z\mathcal G$. In other words, the  tangent space can be identified to $\mathfrak g$ as
\begin{equation}
  \label{eq:153}
  T_z\mathcal G =\{L'_z(a) = z \glaw a \mid a \in \mathfrak g  \}
\end{equation}

Respectively, if we consider $R_z(y)$ as a left group action $ R_z(y) = \Lambda^l( y, z) =y \glaw z $, by differentiation we get a $R'_z : T \mathfrak g \cong  \mathfrak g \rightarrow T_z\mathcal G$ with $\dot\rho (0)=a$ such that
\begin{equation}
  \label{eq:150}
  \lambda^l_{*}(a)(z) = R'_z(a) = \left.\frac{d}{ds}{ \Lambda^l (\rho(s),z)}\right|_{s=0} = a \glaw z
\end{equation}
The map
\begin{equation}
  \label{eq:152}
  \begin{array}{rcl}
  R'_z  : \mathfrak g &\rightarrow& T_z\mathcal G  \\
         a &\mapsto&  a \glaw z
  \end{array}
\end{equation}
determines an isomorphism of $\mathfrak g$ with the tangent space  $T_z\mathcal G$. In other words, the  tangent space can be identified to $\mathfrak g$ as
\begin{equation}
  \label{eq:153}
  T_z\mathcal G =\{R'_z(a) = a \glaw z \mid a \in \mathfrak g  \}
\end{equation}
Any tangent vector $F : \mathcal G \rightarrow T_z\mathcal G$ can be written in either of the forms
\begin{equation}
  \label{eq:155}
  F(z) = L'_z(f(a)) = R'_z(g(z))
\end{equation}
where $f,g \mathcal G \rightarrow \mathfrak g$. 
\subsection{Adjoint representation}
\begin{definition}
Let $p \in \mathcal G$ and let $\sigma (t)$ be a smooth curve on $\mathcal G$ such that $\sigma (0)$ = I and $\dot \sigma(0) = b \in \mathfrak g$. The adjoint representation is defined as
\begin{equation}
\Ad_p(b) =\left. \frac{d}{dt} p\sigma(t)p^{-1}\right|_{t=0}\label{eq:40}
\end{equation}
The derivative of $\Ad$ with respect to the first argument is denoted $\ad$. Let $\rho(s)$ be a smooth curve on $\mathcal G$ such that $\rho(0) = I$  and $\dot \rho(0) = a$, it  yields:
\begin{equation}
  \label{eq:41}
    \ad_a(b) = \left.\frac{d}{ds} \Ad_{\rho(s)}(b)\right|_{s=0}  = [a, b]
\end{equation}
\end{definition}
The adjoint representation can also be expressed with the map
\begin{equation}
  \label{eq:154}
  \Ad_p(b)  = (L_p \glaw R_{p^{-1}})' (b) = (L'_p \glaw R'_{p^{-1}}) (b) = p \glaw b \glaw p^{-1}  
\end{equation}

For a tangent vector given in~\eqref{eq:155}, we have
\begin{equation}
  \label{eq:151}
  g(z) = Ad_z(f(z))
\end{equation}
Another important relation relating $\Ad$, $\ad$ and $\exp$ is
\begin{equation}
  \label{eq:164}
  \Ad_{\exp(a)} =\exp{\ad_a}
\end{equation}


\subsection{Differential of the exponential map} There are multiple ways to represent the differential of $\exp(\cdot)$ at a point $a\in \mathfrak g$. Let us start by the following definition of the differential map at $a\in\mathfrak g$
\begin{equation}
  \label{eq:147}
  \begin{array}{lcl}
    \exp_a' & : & \mathfrak g \rightarrow  T_{exp(a)}\mathcal G\\
            & &  v \mapsto \exp'_a(v)  = \left.\frac{d}{dt} \exp(a+tv)\right|_{t=0}
  \end{array}
\end{equation}
The definition is very similar to the definition of the directional derivative of $\exp$ in the direction $v \in \mathfrak g$ at a point $a\in\mathfrak g$. Using the expression \eqref{eq:153} of the tangent space at $\exp(a)$, we can defined another expression of the differential map denoted as $\dlexp_a : \mathfrak g  \rightarrow \mathfrak g$ such that
\begin{equation}
  \label{eq:156}
  \dlexp_a = L'_{\exp^{-1}(a)} \glaw \exp_a' = L'_{\exp(-a)} \glaw \exp_a' 
\end{equation}
This expression appears as a trivialization of the differential map $\exp'_a$. Using the expression of $L'_z$ in \eqref{eq:152}.
In~\cite[Theorem 2.14.13]{Varadarajan_book1984}, an explicit formula relates $\dlexp_{a}$ to the iteration of the adjoint operator:
\begin{equation}
  \label{eq:43}
  \dlexp_a(b) = \sum_{k=0}^\infty \frac{(-1)^k}{(k+1)!} (\ad_a(b))^k \coloneqq \frac{e - \exp\glaw\ad_a}{\ad_a}(b)
\end{equation}
where $(\ad_a)^k$ is the kth iteration of the adjoint operator:
\begin{equation}
  \label{eq:44}
  \left\{\begin{array}[l]{l}
    (\ad_a)^k(b) = \underbrace{[a, [ a, [ \ldots, a, [ a, b]]]}_{k \text{ times}} \text{ for } k \geq 1 \\
    (\ad_a)^0(b)  = b
  \end{array}\right.
\end{equation}
It is also possible to define the right trivialized differential of the exponential map
\begin{equation}
  \label{eq:162}
  \drexp_a = R'_{\exp^{-1}(a)} \glaw \exp_a' = R'_{\exp(-a)} \glaw \exp_a' 
\end{equation}
that is
\begin{equation}
  \label{eq:163}
  \drexp_a(b) = \exp'_a(b) \glaw \exp(-a)
\end{equation}
With these expression, we have equivalently for 
\begin{equation}
  \label{eq:157}
   \exp_a'(b)  = \exp_a \glaw \dlexp_a(b)\quad \text{ and } \exp_a'(b)  = \drexp_a(b) \glaw   \exp(a)
\end{equation}


To avoid to burden to much the notation, we introduced the unified definition of the differential map  that corresponds to $\dexp=\drexp$ 
\begin{definition}
The differential of the exponential mapping, denoted by $\dexp_a : \mathfrak g \times \mathfrak g \rightarrow \mathfrak g$ is defined as the ``right trivialized'' tangent of the exponential map
\begin{equation}
  \label{eq:42}
  \frac{d}{dt} (\exp(a(t))) = \dexp_{a(t)}(a'(t)) \exp(a(t))
\end{equation}
\end{definition}
An explicit formula relates $\dexp_{a}$ to the iteration of the adjoint operator:
\begin{equation}
  \label{eq:43}
  \dexp_a(b) = \sum_{k=0}^\infty \frac{1}{(k+1)!} (\ad_a(b))^k \coloneqq \frac{\exp\glaw\ad_a-e}{\ad_a}(b)
\end{equation}


\begin{ndrva}
  Say what is not the Jacobian in $\RR^4$
\end{ndrva}

As for $\Ad_a$ and $\ad_a$, the mapping $\dexp_{a}(b)$ is a linear mapping in its second argument for a fixed $a$. Using the relation~\eqref{eq:164}, we can also relate the right and the lest trivialization tangent
\begin{equation}
  \label{eq:165}
\dlexp_a (b) =   (\Ad_{\exp(a)} \glaw \dexp(a))(b) = (\exp(\ad_{-a}) \glaw \frac{e - \exp\glaw\ad_a}{\ad_a})(b) = \frac{e - \exp\glaw\ad_{-a}}{\ad_a}(b) = \dexp_{-a}(b)
\end{equation}
It is also possible to define the  the ``left trivialized'' tangent of the exponential map
\begin{equation}
  \label{eq:46}
   \frac{d}{dt} (\exp(a(t))) =  \exp(a(t)) \dlexp_{a(t)}(a'(t)) = \exp(a(t)) \dexp_{-a(t)}(a'(t)) 
\end{equation}

\begin{ndrva}
  other notation and Lie derivative
  \begin{equation}
    \label{eq:178}
      Df \cdot \widehat \Omega (p) = (\widehat \Omega^r f )(p) 
  \end{equation}
\end{ndrva}



\paragraph{Inverse of the exponential map}


The function $\dexp_{a}$ is an analytical function so it possible to invert it to get
\begin{equation}
  \label{eq:45}
  \dexp^{-1}_{a} = \sum_{k=0}^\infty \frac{B_k}{(k)!} (\ad_a)^k(b) 
\end{equation}
where $B_k$ are the Bernouilli number.

\subsection{Differential of a map $f : \mathcal G \rightarrow \mathfrak g$}

We follow the notation developed in~\cite{Owren.Welfert_BIT2000}. Let us first define the differential of the map $f : \mathcal G \rightarrow \mathfrak g$ as
\begin{equation}
  \label{eq:166}
  \begin{array}[rcl]{rcl}
    f'_z : T_z\mathcal G &\rightarrow&T_{f(z)}\mathfrak g \cong  \mathfrak g\\
    b &\mapsto& \left.\frac{d}{dt} f(z\glaw \exp(t L'_{z^{-1}}(b))) \right|_{t=0}
  \end{array}
\end{equation}
The image of $b$ by $f'_z$   is obtained by first identifying $b$ with an element of $v \in \mathfrak g$ thanks to the left representation of $T_{f(z)}\mathfrak g$ view the left translation map $v= t L'_z(b)$. The exponential mapping transforms $v$ an element $y$ of the Lie Group $\mathcal G$. Then $f'_z$ is obtained by
\begin{equation}
  \label{eq:167}
  f'_z(b) = \lim_{t\rightarrow 0} \frac{f(z\glaw y) - f(z)}{t}
\end{equation}
As we have done for the exponential mapping, it is possible to get a left trivialization of  
\begin{equation}
  \label{eq:169}
  \dd f_z = (f\glaw L_z)' = f'_z \glaw L'_z
\end{equation}
thus
\begin{equation}
  \label{eq:170}
  \dd f_z (a) =  f'_z \glaw L'_z(a) = f'_z(L'_z(a)) =  \left.\frac{d}{dt} f(z\glaw \exp(t a )) \right|_{t=0}
\end{equation}

\paragraph{Newton Method}
Let us imagine that we want to solve $f(y) = 0 $ for $y \in \mathcal G$. A newton method can be written as 
%%% Local Variables:
%%% mode: latex
%%% TeX-master: "DevNotes"
%%% End:

